\documentclass[11pt, a4paper]{article}
\usepackage[margin=0.95in]{geometry}
\usepackage{amsmath}
\usepackage{verbatim}
\usepackage{diagbox}
\usepackage{hyperref}
\usepackage{algorithm}
\usepackage{algorithmic}
\usepackage{listings}
\usepackage{amsfonts}
\usepackage{graphicx}
\usepackage{tikz}
\graphicspath{ {./} }
\usepackage{mathtools}
\DeclarePairedDelimiter{\abs}{\lvert}{\rvert}
\usetikzlibrary{automata,positioning} % automata and positioning libraries are required to use nodes and coordinates in addition to placement propetries.
%\usepackage{parskip}
% equations coulde be %   default number of the equation on the rigth and equation centered 
%   leqno number on the left and equation centered %   fleqn number on the rigth and  equation on the left side
\newcommand{\solution}{ \noindent \textbf{Solution:}}
\newcommand{\problem}[1]{\textit{#1} \medskip}

% get rid of spacing after problem headers
\usepackage{titlesec}
%\titlespacing\subsection{0pt}{12pt plus 4pt minus 2pt}{0pt plus 2pt minus 2pt}
\titlespacing{\subsection}{0pt}{*3}{-\parskip}

\newcommand{\icol}[1]{% inline column vector
  \left(\begin{smallmatrix}#1\end{smallmatrix}\right)%
}

\newcommand{\irow}[1]{% inline row vector
  \begin{smallmatrix}(#1)\end{smallmatrix}%
}

\title{Combinatorial Optimization and Modern Heuristics: Simplex Mini-project}
\author{
	Max Williams\\
    \textsc{Computer Science \& Engineering}\\
	}

\date{\today} 
\begin{document}
\maketitle


\section{Problem Statement}

\subsection{Formal Statement}

% stating the LP problem using equations

A linear programming problem is defined as a set of $n$ inequalities using $m$ variables and an
objective function to be minimized or maximized that is a linear combination of the $m$ variables.
Formally, we state the problem as

\begin{equation*}
\begin{array}{ll@{}ll}
\text{min/maximize}  & \displaystyle\sum\limits_{j=1}^{m} c_{j}x_{j} &&\\
\text{subject to}& \displaystyle\sum\limits_{j=1}^{n}  A_{j,i} & \geq b_j,  &i=1 ,..., m\\
                 &                                                &x_{j} \geq 0, &j=1 ,..., n
\end{array}
\end{equation*}

where $c_n$ is the nth coefficient of the objective functions, $b_n$ is the nth right hand side of
each inequality, and $A_{m,n}$ are the coefficients of mth term of the nth inequality.

\subsection{The Algorithm}

To use the simplex algorithm, we need to rearrange the previous problem statement's inequalities
into equalities by introducing $m$ slack variables.

\begin{equation*}
\begin{array}{ll@{}ll}
\text{min/maximize}  & \displaystyle\sum\limits_{j=1}^{m} c_{j}x_{j} &&\\
\text{subject to}& \displaystyle\sum\limits_{j=1}^{n}  A_{j,i} + s_i &= b_i ,  &i=1 ,..., m\\
                 &                                                &x_{j} \geq 0, &j=1 ,..., n
\end{array}
\end{equation*}

where $s_m$ is the mth slack variable. 

High-level pseudocode for the simplex algorithm to solve this problem is here:

\newcommand{\ass}{$\leftarrow\ $}

\begin{algorithm}
\caption{Simplex Algorithm}
\begin{algorithmic}
    \REQUIRE $A \in \mathbb{R}^{m\times n}, b \in \mathbb{R}^{m}, c \in \mathbb{R}^{n}$
    \STATE {complete \ass false}
    \WHILE{not complete}
        \STATE pivot-col-found, pivot-col \ass find-pivot-col(A)
        \IF{not pivot-col-found}
            \STATE complete \ass true
            \STATE RETURN A[:,-1]
        \ENDIF
        \STATE pivot-row-found, pivot-col \ass find-pivot-row(A, b)
        \IF{not pivot-row-found}
            \STATE ERROR solution not found
        \ENDIF
        \STATE{apply-pivot(A, b, pivot-row, pivot-col)}
    \ENDWHILE
\end{algorithmic}
\end{algorithm}

% describing the Simplex algorithm in pseudocode and equations

\subsection{Implementation}

% describing the source of the Simplex code that you used, including its source code and simple
% examples of how to use it (and on which platform)

I used the \lstinline{linprog} solver from Scipy\footnote{\url{https://www.scipy.org/}}, which
includes an option to use the simplex algorithm. Their simplex implementation can be found on
Github
\footnote{
    \url{https://github.com/scipy/scipy/blob/master/scipy/optimize/\_linprog\_simplex.py}}.
I created a Python script to generate linear programming problems, solve them using this solver, and
record how long it takes to solve problems of various sizes. The results are recorded in Table
\ref{table:time}, showing the average runtime over 10 runs for each (m,n) pair.

Generating solvable LP problems from a given m and n was difficult, and I settled on filling A with
random number from -5 to 5, with additional code to make about half of the values default to 1.
Elements of b and c are all set to 1. This LP problem generation strategy is very inefficient when
m is greater than n, as most of the generated LP problems are not solvable. Addressing this issue
directly was outside of the scope of the project, and instead a trial and error method was used. If
the solver terminated with failure, a new problem was generated until the solver succeeds. This may
invalidate the timing results because it heavily biases the kinds of problems that the solver
solves. The number of attempts to generate 10 valid LP problems is shown in Table \ref{table:tries}.


\section{Experimental Results}

% List the values of m, n, the problem instance statement, the system time needed for solving the LP
% and the solution that you found for each of the above combination of m and n
% Tabulate and graph the time versus m and n

The following linear programming problem was given:

\begin{equation}
    \begin{array}{lcccccccc}
        \text{min }        & x_1  &+& x_2   &+& x_3  &+& x_4 &\\
    \text{subject to } & x_1  & +& 2x_2  &-&  x_3  &-& x_4 & = 1\\
                      & -x_1 & -& 5x_2  &+& 2x_3 &+& 3x_4 & = 1\\
                      &  && && x_i &\geq & 0\  \forall i       &
\end{array}
\end{equation}



Which was converted to the following arrays, and found the solution, $x$. Solving it took 1.925 ms.

\begin{align}
    A &=
    \begin{bmatrix}
        1  &  2  & -1  & -1 \\
        -1  & -5  &  2  &  3 
    \end{bmatrix}\\
    b^T &=
    \begin{bmatrix}
        1  &  1 
    \end{bmatrix}\\
    c &=
    \begin{bmatrix}
        1  &  1  &  1  &  1 
    \end{bmatrix}\\
    x^T &=
    \begin{bmatrix}
        2.0 &  0.0 &  0.0 &  1.0
    \end{bmatrix}
\end{align}


The randomly generated problem statements, solutions, and individual run times are included in the
addendum ``Linear Programming Problem Instances and Solutions''.


\begin{table}[h!]
    \centering
    \begin{tabular}{l|rrrr}
        \backslashbox{n}{m}
        &2 & 6 & 10 & 14\\
        \hline
        4 &  1.547 & 2.198 & 1.829  &  2.256 \\
        10 & 1.477 & 4.884 & 5.726  &  6.863 \\
        20 & 1.631 & 5.602 & 8.634  & 15.813 \\
        30 & 2.500 & 7.932 & 9.726  & 14.775 \\
        40 & 2.464 & 7.472 & 11.317 & 16.587 \\
        50 & 2.286 & 7.680 & 13.277 & 16.673
    \end{tabular}
    \caption{Average time in milliseconds to solve an LP problems for each m,n pair.}
    \label{table:time}
\end{table}


\begin{table}[h!]
    \centering
    \begin{tabular}{l|rrrr}
        \backslashbox{n}{m}
        &2 & 6 & 10 & 14\\
        \hline
        4 & 11 & 98 & 820 & 16208 \\
        10 & 10 & 11 & 334 & 8706 \\
        20 & 10 & 10 & 13 & 96 \\
        30 & 10 & 10 & 10 & 13 \\
        40 & 10 & 10 & 10 & 10 \\
        50 & 10 & 10 & 10 & 10
    \end{tabular}
    \caption{Number of attempts to generate 10 solveable LP problems for each m,n pair.}
    \label{table:tries}
\end{table}

\section{Analysis}

% An analysis of the results: what do you observe as the problem size and constraints vary, etc?

Increasing $n$ and $m$ increases runtime, but not in a consistent way. I suspect this has to do with
the inefficiency of the implementation and the randomness of the generated data. Increasing the
number of constraints to be greater than or equal to the number of variables makes it very difficult
to find solvable LP instances.

\section{Conclusion}

Although it has been superseded by the Interior Points algorithm, the Simplex algorithm is a simple
local search algorithm which works well for smallish matrices and emerges naturally from the
geometric interpretation of linear programming problems.


\newpage
\section{Code Listing}
\lstinputlisting[caption={Code to generate and solve LP problems. Code for outputting tables is
omitted.}, label={lst:solver}, basicstyle=\ttfamily\scriptsize, linerange={1-39}, language=Python, title=solver.py, ]{solver.py}

\newpage
\section{Linear Programming Problem Instances and Solutions}

\tiny
\verbatiminput{matrices.txt}


\end{document}
