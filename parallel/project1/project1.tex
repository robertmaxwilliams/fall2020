\documentclass[12pt,letterpaper]{article}
\usepackage{preamble}

\usepackage[utf8]{inputenc}
\usepackage[english]{babel}
\usepackage{listings}
\usepackage{mathtools}
\usepackage{amsmath}
\usepackage{enumitem}


\newcommand\course{CSE 625}
\newcommand\hwnumber{1}
\newcommand\userID{Max Williams}

\newcommand{\tf}[2]{\frac{\text{#1}}{\text{#2}}}

\begin{document}

\section*{Problem 1}


\section*{Problem 2}

\subsection*{Problem 2.a}
\subsection*{Problem 2.b}
\subsection*{Problem 2.c}

\section*{Problem 3}

\subsection*{Problem 3.a}
\subsection*{Problem 3.b}
\subsection*{Problem 3.c}
\subsection*{Problem 3.d}


\section*{Problem 4}

\lstinputlisting[language=Ant]{a3.asm}

\subsection*{Problem 4.1}

Code segment \textit{(a)} has a true dependency, read after write. \lstinline{R1} is written to upon completion of the
ADD instruction, and the \lstinline{LD} instruction depends on \lstinline{R1} for offsetting the address it loads from.
Once \lstinline{R1} is written to, the \lstinline{LD} instruction is free to run without this hazard.

Code segment \textit{(b)} has no dependencies. The first instruction reads \lstinline{R1} and
\lstinline{R2} and writes \lstinline{R3}. The
second reads \lstinline{R1} and \lstinline{R2} and writes to some location in memory. Thus only ``read after read'' takes
place and there are no data hazards.

Code segment \textit{(c)} has a potential output dependency. If \lstinline{7(R1)} and
\lstinline{200(R7)} happen to point to the same
location in memory, then that location is written to by the first instruction then the second. This
is resolved once the first instruction has finished writing, then the second is free to (possibly)
overwrite it.

Code segment \textit{(d)} creates a branch hazard, and the \lstinline{ST} instruction cannot write to memory until it is
known whether is \lstinline{BEZ} instruction is going to branch. This is because whether or not the
\lstinline{ST}
instruction is ran depends on the result of \lstinline{BEZ}. Thus the \lstinline{ST} must be delayed until after the
branch is finished (possibly) writing to the instruction pointer.

\subsection*{Problem 4.2}

Code segment \textit{(a)} does not allow for out of order execution because the second line depends on the
first. Code segment \textit{(b)} can be reordered without issue. Code segment \textit{(c)} can only be reordered if
\lstinline{7(R1)} and \lstinline{200(R7)} happen to point to different locations. Code segment \textit{(d)} cannot be reordered
because it is unknown whether the \lstinline{ST} instruction will be ran at all. Branch prediction can be used,
for instance by caching the address the \lstinline{ST} instruction is going to use if it is expected to run.

\end{document}
