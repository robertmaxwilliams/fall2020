\documentclass[12pt,letterpaper]{article}
\usepackage{preamble}

\usepackage[utf8]{inputenc}
\usepackage[english]{babel}
\usepackage{listings}
\usepackage{mathtools}
\usepackage{amsmath}
\usepackage{enumitem}


\newcommand\course{CSE 611}
\newcommand\hwnumber{4}
\newcommand\userID{Max Williams}

\newcommand{\tf}[2]{\frac{\text{#1}}{\text{#2}}}

\begin{document}

\section*{Loop Performance}

\subsection*{Introduction}

Choosing the right loop order (or, equivalently, indexing strategy) affects performance greatly. The
cache is able to read and store contiguous chunks of memory, and follows strategies that maximize
performance for contiguous reads. In this paper, I evaluate the 6 different iteration orders for
naive matrix multiplication. In Table \ref{table:perf}, the 6 different orders and their respective
runtime for multiplying two $1000\times 1000$ matrices is listed for two compiler optimization
levels.

\begin{table}[h!]
    \centering
    \begin{tabular}{l| r|r}
        order & time (seconds) -O0 & time (seconds) -O3\\
        \hline
        i-j-k & 4.977  & 1.623\\
        i-k-j & 21.557 & 21.919\\
        k-i-j & 21.501 & 21.821\\
        j-i-k & 7.025  & 2.108\\
        k-j-i & 3.948  & 0.563\\
        j-k-i & 3.907  & 0.368
    \end{tabular}
    \caption{Timings for all six orderings, for two optimization settings: -O0 and -O3.}
    \label{table:perf}
\end{table}

\subsection*{Runtime Difficulty}

The original problem statement called for $5000\times 5000$ matrices, but the program wouldn't
terminate in a reasonable amount of time. This may have been due to the large size overflowing the
largest cache. The program has $3\times 5000 \times 5000 = 75\times 10^6$ floats or 300 million
bytes of memory, which is well below my 8 GB of RAM but still quite large compare to the cache size.
Regardless of the cause, the array size needed to be reduced to $1000 \times 1000$.

\subsection*{Best Performance}

The \lstinline{j-k-i} ordering performed best, and \lstinline{k-j-i} performed almost identically.
The thing to notice here is that both have \lstinline{i} as the innermost loop variable. Looking at
the array access:

\lstinline{C[i+j*N] += A[i+k*N] * B[k+j*N];}

I see that \lstinline{i} only appears as a base index (not multiplied by anything). This means that
within the innermost loop, \lstinline{j} and \lstinline{k} are constant and \lstinline{i} is used to
access contiguous memory.

\subsection*{Worst Performance}

The orders \lstinline{i-k-j} and \lstinline{k-i-j} both have the worst performance. This is because
they both have \lstinline{j} in their innermost loop, which is multiplied by N in the index for both
B and C. This causes the largest number of cache misses, because single floats are being accessed
from memory locations spaced out by 1000 floats. By the time it gets to the next iteration and
\lstinline{j} is back at 0, those locations have already been taken out of cache and cause a cache
miss.

\subsection*{Middlest Performance}

The order \lstinline{i-j-k} and \lstinline{j-i-k} both have middling performance. This is mostly due
to \lstinline{k} being in the innermost loop, which only causes one spaced-out read in the innermost
loop.

\subsection*{Optimization Level}

For the best performing loops, compiler optimization makes a big improvement, but for the worst
performing loops, it actually makes them slightly worse. 

\section*{Conclusion}

Iteration order optimization can make a huge difference in performance - nearly sixty times in the
most severe case. This means that when developing performance critical sections of code, careful
consideration of iteration order should be made or experimental testing should be done.  Based on
these results, I conclude that the following rule is useful when deciding iteration order: ``The
innermost loop is most important, and having the innermost loop variable multiplied by anything will
harm performance.'' 

Of course, premature optimization should be avoided - profile before optimizing, and only optimize
if it makes a difference for your product. Doing this kind of optimization can lead to code which is
harder to read then putting your loops in their logical, intuitive order.


\section*{Code Listing}

\lstinputlisting{six-loops.c}





\end{document}
